\documentclass[aspectratio=1610]{beamer}
\usepackage{graphicx}
\usepackage{amssymb,amsmath,amsfonts}
\usepackage[spanish,es-tabla]{babel}
\usepackage[ansinew]{inputenc}
\usepackage{xcolor}
%\usepackage{enumitem}
\usetheme{Madrid} 
\useinnertheme{circles}
\usecolortheme{default} 

\setbeamercolor{structure}{fg=green!60!black} % Color principal a un tono de verde
\setbeamercolor{title}{fg=green!60!black} % Establece el color del t�tulo al verde
\setbeamercolor{subtitle}{fg=green!50!black} % Establece el color del subt�tulo al verde

\usefonttheme[onlymath]{serif} % Cambiar a fuente serif para matem�ticas
%\usefonttheme{serif}

\title{\bf Regresi�n Lineal}
\subtitle{Consideraciones para el an�lisis de datos en los cursos de laboratorio de F�sica}
\author{H�ctor F. Hern�ndez G.}
\date{\today}

% Definir el color verde personalizado
\definecolor{customgreen}{RGB}{0,128,0}

% Configurar los colores de Beamer para el tema Warsaw
\setbeamercolor{title in head/foot}{bg=customgreen, fg=white}
\setbeamercolor{author in head/foot}{bg=customgreen, fg=white}
\setbeamercolor{date in head/foot}{bg=customgreen, fg=white}
\setbeamercolor{section in head/foot}{bg=customgreen, fg=white}
\setbeamercolor{subsection in head/foot}{bg=customgreen, fg=white}
\setbeamercolor{frametitle}{bg=customgreen, fg=white}
\setbeamercolor{block title}{bg=customgreen, fg=white}
\setbeamercolor{block body}{bg=customgreen!10, fg=black}
\setbeamercolor{item}{fg=customgreen}


\begin{document}
\frame{\titlepage}


\begin{frame}
\frametitle{Resumen}
En esta presentaci�n se describen los resultados del experimento sobre [tema del experimento]. Se discutir�n los m�todos utilizados y se analizar�n los resultados obtenidos, compar�ndolos con las teor�as existentes y discutiendo las posibles fuentes de error.
\end{frame}

\begin{frame}
\frametitle{Introducci�n}
\lipsum[1]
\end{frame}

\begin{frame}
\frametitle{Metodolog�a}
\framesubtitle{Materiales}
\begin{itemize}
    \item Material 1
    \item Material 2
    \item Material 3
\end{itemize}
\end{frame}

\begin{frame}
\frametitle{Metodolog�a}
\framesubtitle{Procedimiento}
\lipsum[3]
\end{frame}

\begin{frame}
\frametitle{Experimento y Resultados}
\framesubtitle{Descripci�n del Experimento}
\lipsum[4]
\end{frame}

\begin{frame}
\frametitle{Datos y Resultados}
\begin{table}[h]
\centering
\caption{Resultados obtenidos en el experimento.}
\begin{tabular}{|c|c|c|}
\hline
\textbf{Medici�n} & \textbf{Valor} & \textbf{Incertidumbre} \\
\hline
1 & 10.5 & 0.1 \\
2 & 10.7 & 0.1 \\
3 & 10.6 & 0.1 \\
\hline
\end{tabular}
\end{table}
\end{frame}

\begin{frame}
\frametitle{Gr�ficos}
\begin{figure}[h]
\centering
\includegraphics[width=0.7\textwidth]{grafico.png}
\caption{Gr�fico de los resultados obtenidos.}
\end{figure}
\end{frame}

\begin{frame}
\frametitle{Discusi�n y Conclusiones}
\lipsum[5]
\end{frame}

\begin{frame}
\frametitle{Referencias}
\begin{itemize}
    \item Autor, A. (A�o). T�tulo del libro. Editorial.
    \item Autor, B. (A�o). T�tulo del art�culo. \textit{Nombre de la Revista}, volumen(n�mero), p�ginas.
    \item \url{https://www.example.com}
\end{itemize}
\end{frame}


\end{document}