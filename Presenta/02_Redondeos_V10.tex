\documentclass[aspectratio=1610]{beamer}
\usepackage{graphicx}
\usepackage{amssymb,amsmath,amsfonts}
\usepackage[spanish,es-tabla]{babel}
\usepackage[ansinew]{inputenc}
\usepackage{xcolor}
%\usepackage{enumitem}
\usetheme{Madrid} 
\useinnertheme{circles}
\usecolortheme{default} 

\setbeamercolor{structure}{fg=green!60!black} % Color principal a un tono de verde
\setbeamercolor{title}{fg=green!60!black} % Establece el color del t�tulo al verde
\setbeamercolor{subtitle}{fg=green!50!black} % Establece el color del subt�tulo al verde

\usefonttheme[onlymath]{serif} % Cambiar a fuente serif para matem�ticas
%\usefonttheme{serif}

\title{\bf Regresi�n Lineal}
\subtitle{Consideraciones para el an�lisis de datos en los cursos de laboratorio de F�sica}
\author{H�ctor F. Hern�ndez G.}
\date{\today}

% Definir el color verde personalizado
\definecolor{customgreen}{RGB}{0,128,0}

% Configurar los colores de Beamer para el tema Warsaw
\setbeamercolor{title in head/foot}{bg=customgreen, fg=white}
\setbeamercolor{author in head/foot}{bg=customgreen, fg=white}
\setbeamercolor{date in head/foot}{bg=customgreen, fg=white}
\setbeamercolor{section in head/foot}{bg=customgreen, fg=white}
\setbeamercolor{subsection in head/foot}{bg=customgreen, fg=white}
\setbeamercolor{frametitle}{bg=customgreen, fg=white}
\setbeamercolor{block title}{bg=customgreen, fg=white}
\setbeamercolor{block body}{bg=customgreen!10, fg=black}
\setbeamercolor{item}{fg=customgreen}


\begin{document}
\frame{\titlepage}

\begin{frame}
\frametitle{Introducci�n}
En este cap�tulo, se abordan conceptos fundamentales relacionados con el manejo de datos num�ricos en el laboratorio. 
\begin{itemize}
    \item Redondeo de n�meros para obtener valores aproximados.
    \item Consideraci�n de cifras significativas para expresar la precisi�n de una magnitud.
\end{itemize}
\end{frame}

\begin{frame}
\frametitle{Redondeo}
El redondeo permite simplificar n�meros manteniendo una precisi�n adecuada.
\begin{itemize}
    \item Si el primer d�gito a suprimir es menor que 5, el �ltimo d�gito conservado no cambia.
    \item Si es mayor o igual a 5, se aumenta en uno el �ltimo d�gito conservado.
\end{itemize}
\begin{table}[]
    \centering
    \begin{tabular}{ccccccc}
        \hline
        & N�mero & 5 cifras & 4 cifras & 3 cifras & 2 cifras & 1 cifra \\ \hline
        a & 3.14159 & 3.1416 & 3.142 & 3.14 & 3.1 & 3 \\
        b & 9.8070e-3 & 9.8070e-3 & 9.807e-3 & 9.81e-3 & 9.8e-3 & 1e-2 \\
        c & 0.644510 & 0.64451 & 0.6445 & 0.645 & 0.64 & 0.6 \\
        d & 327508 & 32751e1 & 3275e2 & 328e3 & 33e4 & 3e5 \\ \hline
    \end{tabular}
\end{table}
\end{frame}

\begin{frame}
\frametitle{Cifras Significativas}
Las cifras significativas son los d�gitos que aportan informaci�n sobre la precisi�n de una medici�n.
\begin{itemize}
    \item Se cuentan desde el primer d�gito diferente de cero.
    \item Incluyen todos los d�gitos hasta el primer d�gito afectado por el error.
\end{itemize}
\textbf{Ejemplos:}
\begin{enumerate}
    \item 1.231 m; 123.1 cm; 1231 mm (4 cifras significativas)
    \item 21.03 g y 200.3 cm (4 cifras significativas)
    \item 2.00 cm y 740 m (3 cifras significativas)
    \item 0.48 s y 0.0052 g (2 cifras significativas)
    \item 323e-3 kg y 3.00e8 m/s (3 cifras significativas)
\end{enumerate}
\end{frame}

\begin{frame}
\frametitle{Operaciones con Cifras Significativas}
\textbf{Suma y Resta}
\begin{itemize}
    \item Redondear el resultado final al n�mero de cifras decimales de la magnitud menos precisa.
\end{itemize}
\begin{table}[]
    \centering
    \begin{tabular}{ccc}
        \text{1)} 25.340 & \text{2)} 58.0 & \text{3)} 1.6523 \\
        +5.465 & +0.038 & -0.015 \\
        +0.322 & +1.0001 & \\
        \hline
        31.127 & 59.0381 & 1.6373 \\
        \hline
        Resultado & 59.0 & 1.637
    \end{tabular}
\end{table}
\end{frame}

\begin{frame}
\frametitle{Operaciones con Cifras Significativas}
\textbf{Multiplicaci�n y Divisi�n}
\begin{itemize}
    \item Redondear el resultado final al n�mero de cifras significativas del factor menos preciso.
\end{itemize}
\begin{enumerate}
    \item $7.485 \cdot 8.61 = 64.4 \text{m}^2$
    \item $7485 \cdot 8.61 = 644 \times 10^2 \text{m}^2$
    \item $\dfrac{0.1342}{1.54} = 0.0871$
\end{enumerate}
\end{frame}

\begin{frame}
\frametitle{Orden de Magnitud}
El orden de magnitud de una cantidad es la potencia de diez m�s cercana a esa cantidad.
\begin{itemize}
    \item La masa de la Tierra: $5.983 \times 10^{24}$ kg $\rightarrow 10^{25}$ kg
    \item 0.0035 $\rightarrow 10^{-3}$
    \item $800 \times 10^{-3}$ $\rightarrow 10^0$ cm
\end{itemize}
\end{frame}

\begin{frame}
\frametitle{Ejemplos de Orden de Magnitud}
Determinar el orden de magnitud de una cantidad f�sica:
$$
10^{24} \text{kg} < 5.983 \times 10^{24} \text{kg} < 10^{25} \text{kg}
$$
Diferencias respectivas:
$$
5.983 \times 10^{24} \text{kg} - 10^{24} \text{kg} = 4.983 \times 10^{24} \text{kg}
$$
$$
10^{25} \text{kg} - 5.983 \times 10^{24} \text{kg} = 4.017 \times 10^{24} \text{kg}
$$
La menor diferencia indica mayor proximidad, por lo tanto, $10^{25}$ es el orden de magnitud m�s cercano.
\end{frame}

\begin{frame}
\frametitle{Ejercicios}
\begin{enumerate}
    \item Determinar el n�mero de cifras significativas y el orden de magnitud de:
    \begin{enumerate}
        \item Radio de la Tierra: $6.371 \times 10^6$ m
        \item Volumen de la Tierra: $1.087 \times 10^{21}$ m$^3$
        \item Aceleraci�n de la gravedad: $9.80665$ m/s$^2$
    \end{enumerate}
    \item Expresar en �rdenes de magnitud:
    \begin{enumerate}
        \item Edad del universo: $3 \times 10^{10}$ a�os en segundos.
        \item Velocidad de la luz en: m/s, m/h y km/d�a.
        \item Densidad del hierro en: kg/m$^3$ y g/cm$^3$.
    \end{enumerate}
    \item Un estudiante midi� el radio de una esfera como 10.00 mm.
    \begin{enumerate}
        \item Calcular el �rea.
        \item Calcular el volumen.
        \item Indicar el orden de magnitud.
    \end{enumerate}
    \item Expresar los siguientes n�meros con diferentes cifras significativas:
    \begin{table}[]
        \centering
        \begin{tabular}{ccc}
            \hline
            0.4536 & 98.372 & 70045.6 \\
            163571 & 3.13100 & 26.39 \\
            0.45330 & 0.00332998 & 20150.0 \\ \hline
        \end{tabular}
    \end{table}
    \item Calcular $L$ con cifras significativas:
    $$
    L = \frac{k^2}{a} + a
    $$
    Donde $k = 14.3$ cm y $a = 853$ mm.
\end{enumerate}
\end{frame}

\end{document}