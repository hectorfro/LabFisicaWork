\chapter{�C\'omo escribir un informe de laboratorio?}

\section{Introducci\'on}

La \'ultima secci\'on se enfocar\'a en las pautas y estructura para la redacci\'on de informes de laboratorio. Se proporcionar\'an consejos sobre la organizaci\'on de la informaci\'on, la inclusi\'on de resultados, tablas y gr\'aficos, la elaboraci\'on de conclusiones y la citaci\'on adecuada de fuentes.




Describir en un texto el trabajo de una investigaci\'on para que sea comprensible para los dem\'as y aceptable para su publicaci\'on es una tarea que puede resultar  dif\'icil y escabrosa. Por eso es necesario que los estudiantes universitarios aprendan a implicarse en esta tarea desde sus primeros contactos con los laboratorios. 

En este contexto universitario el  informe o monograf\'ia de laboratorio es un texto que se basa en las experiencias obtenidas en el laboratorio, o de una investigaci\'on te\'orica, que pretender ser una primera aproximaci\'on a las publicaciones cient\'ificas profesionales. 

La idea principal es que el estudiante y futuro investigador comunique sus resultados, logros y avances  de la mejor manera posible, con la suficiente precisi\'on y claridad como para que otros investigadores o estudiantes puedan interpretar tambi\'en los resultados, incluso replicar la experiencia. 

Por lo tanto, es importante que el estudiante comience a familiarizarse con el m\'etodo cient\'ifico y que sea capaz, no solo  de relacionar la teor\'ia con los resultados experimentales, sino que tambi\'en sea capaz de comunicar su experiencia de manera satisfactoria. 

\section{Estructura}
Un reporte, informe o monograf\'ia debe estar conformado por  conjunto de partes o secciones que son esenciales para su elaboraci\'on. B\'asicamente son las que se muestra a continuaci\'on:
\begin{enumerate}
\item Resumen: El resumen debe tener el contenido suficiente para que el lector se construya una idea completa de qu\'e se hizo y cu\'ales fueron los resultados. 

\item Introducci\'on: Aqu\'i se explica la raz\'on de ser del art\'iculo y su contexto. Se debe describir el problema, los antecedentes y la justificaci\'on del trabajo. 

\item Metodolog\'ia: Aqu\'i se hace referencia a los recursos materiales que se utilizaron en el trabajo, las t\'ecnicas utilizadas,  la descripci\'on de c\'omo se realiz\'o el trabajo. Esta secci\'on  es importante ya que debe garantizar que otra persona que lea el trabajo  pueda replicar el experimento o los c\'alculos que se realizaron. 

\item Experimento y resultados: se exponen los datos obtenidos, los c\'alculos que se realizaron. Tambi\'en se puede asomar algunos avances sobre las ideas que se discutir\'an con mayor profundidad en la conclusiones pero sin entrar en an\'alisis num\'ericos. 

\item Discusi\'on y conclusiones: Se hace un an\'alisis de los resultados obtenidos y de su relevancia, se pueden hacer comparaciones con otros trabajos, hacer predicciones, comprobar objetivos. 

\item Referencias: se escribe la lista completa de las referencias que se citaron en el texto. 
Es recomendable apegarse a un est\'andar de referencias ya que los estilos pueden variar dependiendo del \'area cient\'ifica o el tipo de publicaci\'on. 
\end{enumerate}


\section{Modelo para un informe de laboratorio}

A continuaci\'on y a manera de ap\'endice se anexa un modelo de informe de laboratorio para ser completado con la informaci\'on correspondiente. 

La fuente en \LaTeX{} del documento\footnote{Art\'iculo realizado por el Prof Luis A. N\'u\~nez de la Escuela de F\'isica} la pueden encontrar en \url{https://github.com/hectorfro/CursosUIS/tree/main/LabFisII23B}



