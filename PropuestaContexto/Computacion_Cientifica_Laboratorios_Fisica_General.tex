\documentclass{article}
\usepackage{amsmath}
\usepackage{graphicx}
\usepackage{hyperref}

\title{Computación Científica para los Laboratorios de Física General}
\author{}
\date{}

\begin{document}

\maketitle

\section*{Objetivo del Curso}
Desarrollar en los estudiantes habilidades de pensamiento computacional y manejo de herramientas computacionales para el análisis de datos experimentales, con aplicaciones específicas en laboratorios de física.

\section*{Competencias}
Al finalizar el curso, los estudiantes serán capaces de:
\begin{enumerate}
    \item Aplicar conceptos básicos de computación científica en la resolución de problemas experimentales.
    \item Utilizar software de computación científica y la biblioteca de Python para análisis de datos y visualización.
    \item Implementar técnicas de análisis de errores y propagación de incertidumbres en cálculos científicos.
    \item Realizar ajustes de curvas lineales y no lineales a datos experimentales.
    \item Aplicar métodos de análisis espectral y Monte Carlo en contextos físicos.
    \item Resolver ecuaciones diferenciales aplicadas a fenómenos físicos usando métodos computacionales.
    \item Redactar informes de laboratorio claros y precisos utilizando LaTeX.
\end{enumerate}

\section*{Distribución por Semanas}

\subsection*{Semana 1-2: Introducción y Fundamentos}
\begin{enumerate}
    \item \textbf{Ciencia computacional básica}
    \begin{itemize}
        \item Introducción a la computación científica.
        \item Importancia en los laboratorios de física.
        \item Conceptos fundamentales y aplicaciones.
    \end{itemize}
    \item \textbf{Software computacional básico}
    \begin{itemize}
        \item Introducción a herramientas y software: Python, Jupyter Notebooks.
        \item Instalación y configuración de entornos de trabajo.
        \item Rudimentos sobre Linux
    \end{itemize}
\end{enumerate}

\subsection*{Semana 3-4: Herramientas de Python para Computación Científica}
\begin{enumerate}
    \item \textbf{El ecosistema Python: numpy, sympy, matplotlib, scipy}
    \begin{itemize}
        \item Uso de numpy para cálculos numéricos.
        \item sympy para cálculos simbólicos.
        \item matplotlib para visualización de datos.
        \item scipy para análisis científicos avanzados.
    \end{itemize}
    \item \textbf{LaTeX}
    \begin{itemize}
        \item Introducción a LaTeX.
        \item Redacción de informes científicos.
        \item Incorporación de ecuaciones, figuras y tablas.
    \end{itemize}
\end{enumerate}

\subsection*{Semana 5-6: Fundamentos de Mediciones y Errores}
\begin{enumerate}
    \item \textbf{Mediciones en el laboratorio}
    \begin{itemize}
        \item Tipos de mediciones y técnicas.
        \item Prácticas de buenas mediciones.
    \end{itemize}
    \item \textbf{Redondeo, cifras significativas y orden de magnitud}
    \begin{itemize}
        \item Reglas de redondeo y su importancia.
        \item Uso correcto de cifras significativas.
    \end{itemize}
    \item \textbf{Teoría de Errores}
    \begin{itemize}
        \item Tipos de errores: sistemáticos y aleatorios.
        \item Propagación de errores.
        \item Introducción a la teoría de incertidumbres.
    \end{itemize}
\end{enumerate}

\subsection*{Semana 7-8: Análisis de Datos y Representación Gráfica}
\begin{enumerate}
    \item \textbf{Representación de los datos y el análisis gráfico}
    \begin{itemize}
        \item Técnicas de visualización de datos.
        \item Gráficos de dispersión, histogramas, y gráficos de barras.
    \end{itemize}
    \item \textbf{Análisis de datos: mínimos cuadrados}
    \begin{itemize}
        \item Métodos de mínimos cuadrados para ajustes lineales.
        \item Evaluación de la calidad del ajuste.
    \end{itemize}
    \item \textbf{Herramientas de vizualización}
    \begin{itemize}
        \item Herramientas avanzadas de visualización.
        \item Creación de gráficos interactivos.
    \end{itemize}
\end{enumerate}

\subsection*{Semana 9-10: Ajuste de Curvas y Análisis Espectral}
\begin{enumerate}
    \item \textbf{Ajustes lineales y  cuadráticos}
    \begin{itemize}
        \item Ajuste de curvas lineales y cuadráticas.
        \item Análisis de resultados y aplicaciones.
    \end{itemize}
    \item \textbf{ajustes no lineales }
    \begin{itemize}
        \item Métodos de ajuste no lineal.
        \item Aplicaciones en experimentos de física.
    \end{itemize}
    \item \textbf{Analísis de Fourier }
    \begin{itemize}
        \item Introducción al análisis de Fourier.
        \item Aplicaciones en la física de ondas y señales.
    \end{itemize}
\end{enumerate}

\subsection*{Semana 11-12: Métodos Avanzados de Análisis}
\begin{enumerate}
    \item \textbf{Monte Carlo: Aleatoriedad, paseos y decaimiento}
    \begin{itemize}
        \item Simulaciones de Monte Carlo.
        \item Aplicaciones en decaimiento radiactivo y caminatas aleatorias.
    \end{itemize}
    \item \textbf{Cálculos con Matrices}
    \begin{itemize}
        \item Operaciones matriciales y sus aplicaciones.
        \item Uso de matrices en problemas físicos.
    \end{itemize}
\end{enumerate}

\subsection*{Semana 13: Ecuaciones Diferenciales}
\begin{enumerate}
    \item \textbf{Resolviendo ecuaciones diferenciales: Oscilaciones no lineales}
    \begin{itemize}
        \item Solución numérica de ecuaciones diferenciales.
        \item Aplicaciones en oscilaciones no lineales.
    \end{itemize}
\end{enumerate}


\subsection*{Semana 14: Redacción de Informes}
\begin{enumerate}
    \item \textbf{Cómo escribir un informe de laboratorio}
    \begin{itemize}
        \item Estructura de un informe científico.
        \item Buenas prácticas en la redacción de informes.
    \end{itemize}
\end{enumerate}


\section*{Evaluación}
\begin{itemize}
    \item \textbf{Tareas semanales}: 40\%
    \item \textbf{Proyecto final}: 30\%
    \item \textbf{Exámenes (intermedio y final)}: 20\%
    \item \textbf{Participación y asistencia}: 10\%
\end{itemize}

\section*{Materiales y Recursos}
\begin{itemize}
    \item \textbf{Software}: Python, Jupyter Notebooks, LaTeX.
    \item \textbf{Libros y artículos}: Material de referencia proporcionado por el instructor.
\end{itemize}

\end{document}
